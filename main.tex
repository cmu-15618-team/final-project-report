\documentclass[10pt,sigconf,camera-ready]{acmart}

\settopmatter{printacmref=false, printccs=false, printfolios=true}

\usepackage{booktabs} % For formal tables

\graphicspath{{figure/}{figures/}}

% Copyright
% \setcopyright{none}
\setcopyright{acmcopyright}
%\setcopyright{acmlicensed}
% \setcopyright{rightsretained}
%\setcopyright{usgov}
%\setcopyright{usgovmixed}
%\setcopyright{cagov}
%\setcopyright{cagovmixed}


% DOI
% \acmDOI{10.475/123_4}

% ISBN
% \acmISBN{123-4567-24-567/08/06}

%Conference
% \acmConference[SHORTNAME'23]{ACM Long Conference Name conference}{July 1997}{City, State, Country} 
% \acmYear{2023}
% \copyrightyear{2023}

% \acmPrice{15.00}


\begin{document}
\title{Reevaluating In-Memory Parallel Hash Join Designs}
\titlenote{Course final project of Spring 2024 15-618 parallel programming course at Carnegie Mellon University. Do not distribute.}
% \subtitle{Paper \#, XXX pages}

\author{Zhidong Guo}
% \authornote{Note}
% \orcid{1234-5678-9012}
\affiliation{%
  \institution{Carnegie Mellon University}
  \streetaddress{5000 Forbes Avenue}
  \city{Pittsburgh} 
  \state{PA} 
  \postcode{15213}
}
\email{zhidongg@andrew.cmu.edu}

\author{Ye Yuan}
% \orcid{1234-5678-9012}
\affiliation{%
  \institution{Carnegie Mellon University}
  \streetaddress{5000 Forbes Avenue}
  \city{Pittsburgh} 
  \state{PA} 
  \postcode{15213}
}
\email{yeyuan3@andrew.cmu.edu}


% The default list of authors is too long for headers}
\renewcommand{\shortauthors}{Z. Guo et al.}


\begin{abstract}
This paper provides a sample of a \LaTeX\ document which conforms,
somewhat loosely, to the formatting guidelines for
ACM SIG Proceedings.\footnote{This is an abstract footnote}
\end{abstract}

%
% The code below should be generated by the tool at
% http://dl.acm.org/ccs.cfm
% Please copy and paste the code instead of the example below. 
%
\begin{CCSXML}
<ccs2012>
 <concept>
  <concept_id>10010520.10010553.10010562</concept_id>
  <concept_desc>Computer systems organization~Embedded systems</concept_desc>
  <concept_significance>500</concept_significance>
 </concept>
 <concept>
  <concept_id>10010520.10010575.10010755</concept_id>
  <concept_desc>Computer systems organization~Redundancy</concept_desc>
  <concept_significance>300</concept_significance>
 </concept>
 <concept>
  <concept_id>10010520.10010553.10010554</concept_id>
  <concept_desc>Computer systems organization~Robotics</concept_desc>
  <concept_significance>100</concept_significance>
 </concept>
 <concept>
  <concept_id>10003033.10003083.10003095</concept_id>
  <concept_desc>Networks~Network reliability</concept_desc>
  <concept_significance>100</concept_significance>
 </concept>
</ccs2012>  
\end{CCSXML}

\ccsdesc[500]{Computer systems organization~Embedded systems}
\ccsdesc[300]{Computer systems organization~Redundancy}
\ccsdesc{Computer systems organization~Robotics}
\ccsdesc[100]{Networks~Network reliability}

% We no longer use \terms command
%\terms{Theory}

\keywords{database, hash join, multi-core, performance}


\maketitle

\section{Background}


\section{Approach}


\section{Experiment}


\section{Conclusion}


\bibliographystyle{acm}
\bibliography{sigproc} 

\end{document}
